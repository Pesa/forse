\chapter{Requisiti}
\begin{enumerate}
\item Il sistema deve essere composto da più entità concorrenti e distribuite su una rete.
\item La simulazione evolve in modo deterministico.
\item Il circuito è suddiviso in settori ed è configurabile dall'utente tramite file di configurazione. L'utente deve poter specificare le seguenti informazioni:
        \begin{itemize}
        \item tipo di settore (rettilineo, curvilineo, entrata e uscita dai \textit{box}, intermedio cronometrico, traguardo);
        \item lunghezza e larghezza di ciascun settore;
        \item se il settore è una curva, direzione (destra o sinistra) e raggio di curvatura;
        \item pendenza del suolo;
        \item tempo atmosferico iniziale;
        \item eventuali variazioni del tempo atmosferico nel corso della gara.
        \end{itemize}
\item Insieme configurabile di concorrenti aventi le seguenti caratteristiche:
        \begin{itemize}
        \item nome;
        \item scuderia di appartenenza;
        \item esperienza/bravura;
        \item peso;
        \item vettura utilizzata.
        \end{itemize}
\item Parametri di configurazione delle auto:
        \begin{itemize}
        \item capienza del serbatoio;
        \item quantità di carburante presente nel serbatoio ad inizio gara;
        \item tipo dei pneumatici montati ad inizio gara (\textit{slick}, intermedi, da bagnato);
        \item potenza del motore;
        \item potenza dei freni;
        \item peso a secco.
        \end{itemize}
\item Parametri di configurazione di una competizione:
        \begin{itemize}
        \item numero totale di giri da effettuare;
        \item opzionalmente, la posizione dei concorrenti sulla griglia di partenza;
        \end{itemize}
\item Il sistema di controllo deve essere composto da:
        \begin{itemize}
        \item un pannello generale indicante posizione dei concorrenti in gara, tempi di percorrenza attuali e miglior tempo sul giro, con la possibilità per l'utente di sospendere e riprendere la simulazione;
        \item situazione atmosferica della pista, con la possibilità per l'utente di apportare modifiche alle condizioni meteo anche dopo che la gara è iniziata;
        \item un pannello per ciascuna scuderia che riporta i parametri tecnici rilevanti delle vetture dei propri piloti (carburante residuo, mescola dei pneumatici e loro condizioni di usura) nonché i tempi di percorrenza per settore di pista e totali; deve essere inoltre possibile per l'utente forzare la sosta ai box e il ritiro di un'auto.
        \end{itemize}
\end{enumerate}
 
