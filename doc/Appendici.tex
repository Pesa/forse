\chapter{File di configurazione}

\section*{Team e Car}
Le impostazioni delle scuderie, delle auto e dei piloti sono contenuti in uno stesso file di configurazione.

\begin{lstlisting}
[
        {team_name, "McLaren"},
        {brake, 34000},
        {power, 25000},
        {weight, 650},
        {cars, [
                [{id, 2},
                {name, "Lewis Hamilton"},
                {skill, 6},
                {weight, 68},
                {fuel, 70},
                {tyres, slick}],
                [{id, 6},
                {name, "Heikki Kovalainen"},
                {skill, 4},
                {weight, 62},
                {fuel, 80},
                {tyres, slick}]
                ]}
].
\end{lstlisting}

I parametri di configurazione sono descritti in \Erlang{} per risparmiare tempo sfruttando le funzioni di \textit{parsing} offerte dal linguaggio. Si tratta di una lista di coppie chiave-valore la cui semantica è spiegata nella seguente tabella:

\begin{center}
\begin{tabular}{|p{0.15\textwidth}|p{0.75\textwidth}|}
\hline
\multicolumn{2}{|c|}{Scuderia e Auto}\\
\hline
\texttt{team\_name} & Nome della scuderia\\
\texttt{brake} & Potenza dei freni\\
\texttt{power} & Potenza del motore\\
\texttt{weight} & Peso dell'auto a secco\\
\texttt{cars} & Lista dei piloti e della configurazione iniziale delle auto\\
\hline
\multicolumn{2}{|c|}{Pilota e Auto}\\
\hline
\texttt{id} & Posizione nella griglia di partenza\\
\texttt{name} & Nome del pilota\\
\texttt{skill} & Abilità del pilota (intero tra 1 e 10 compresi)\\
\texttt{weight} & Peso del pilota\\
\texttt{fuel} & Carburante presente nell'auto ad inizio gara\\
\texttt{tyres} & Tipo di gomme montate sull'auto ad inizio gara\\
\hline
\end{tabular}
\end{center}

L'unico parametro opzionale tra quelli appena elencati è \texttt{id}. Nel caso in cui nel file di configurazione esista almeno un pilota in cui il parametro \texttt{id} è omesso allora l'ordine nella griglia di partenza di tutti i concorrenti sarà generato casualmente.

\section*{Track}
Il file di configurazione di \texttt{track} contiene le informazioni riguardanti la conformazione della pista espressa in settori. La configurazione è composta da una lista di tuple, ciascuna delle quali rappresenta un settore.

Di seguito la sintassi per descrivere i tipi di settore possibili e i parametri consentiti.

\begin{itemize}
\item \verb!{straight, LENGTH, MIN_LANE, MAX_LANE, INCL, RAIN}!\\ Rettilineo
\item \verb!{right, LENGTH, RADIUS, MIN_LANE, MAX_LANE, INCL, RAIN}!\\ Curva a destra
\item \verb!{left, LENGTH, RADIUS, MIN_LANE, MAX_LANE, INCL, RAIN}!\\ Curva a sinistra
\item \verb!{finish_line}!\\ Traguardo
\item \verb!{intermediate}!\\ Intermedio cronometrico
\item \verb!{pitlane_entrance}!\\ Inizio della zona dei \textit{box}
\item \verb!{pitlane_exit}!\\ Fine della zona dei \textit{box}
\end{itemize}

\begin{tabularx}{\textwidth}{|l|X|}
\hline
\texttt{LENGTH} & Lunghezza del settore in metri, possibilmente un valore multiplo di 5\\
\hline
\texttt{RADIUS} & Raggio di curvatura in metri\\
\hline
\texttt{MIN\_LANE} & Minimo indice di corsia consentito, intero positivo\\
\hline
\texttt{MAX\_LANE} & Massimo indice di corsia consentito, intero positivo\\
\hline
\texttt{INCL} & Inclinazione della pista, deve appartenere all'intervallo $(-90^\circ, +90^\circ)$\\
\hline
\texttt{RAIN} & Condizioni atmosferiche all'inizio della gara, intero tra 0 e 10 inclusi\\
\hline
\end{tabularx}

\section*{Weather}
Questo file contiene una lista di tuple, ciascuna delle quali rappresenta le variazioni del tempo atmosferico in un istante di gara.

Le tuple sono nella forma:
\[ \{\mbox{\texttt{WHEN}},\;\: [\{\mbox{\texttt{WHERE}},\; \mbox{\texttt{RAIN}}\},\; \ldots\;\: ]\} \]
\begin{itemize}
\item \texttt{WHEN}: quando, rispetto al tempo di gara, avverrà il cambiamento. Può essere espresso in secondi oppure nel formato $\{h,m,s\}$.
\item \texttt{WHERE}: in che settore avverrà il cambiamento, l'indice si riferisce alla posizione del settore nel file di configurazione della pista.
\item \texttt{RAIN}: quantità di pioggia presente nel settore, un intero nell'intervallo $[0, 10]$.
\end{itemize}

\section*{config.hrl}
Il prototipo consente di modificare delle costanti utilizzate per la simulazione.
Queste costanti sono raggruppate nel file \texttt{include/config.hrl} e una loro modifica richiede la ricompilazione del prototipo per avere effetto.

La maggior parte delle costanti presenti in questo file agisce direttamente sul modello fisico della simulazione e sulla rappresentazione interna della pista per ottenere una simulazione a grana più o meno fine. Una delle costanti più importanti è infatti la lunghezza dei segmenti in cui viene diviso il tracciato, attualmente impostata a 5 metri.

Questi valori possono essere modificati, tuttavia tale operazione è sconsigliata poiché si potrebbe facilmente dare luogo a situazioni limite nelle quali, per esempio, nessuna auto riuscirebbe a percorrere nemmeno un tratto senza uscire di pista.

Inoltre, la validità di tali valori non viene controllata dal prototipo, di conseguenza potrebbero addirittura essere sollevate eccezioni durante i calcoli (divisione per zero, estrazione di radice quadrata di un numero negativo, e così via).


\chapter{Avvio del prototipo}
Per facilitare l'utilizzo del prototipo vengono forniti degli \textit{script} di avvio.

Si noti che le seguenti componenti possono essere presenti nel sistema in istanze multiple ed avviate anche quando la simulazione è già in corso: \texttt{race\_info}, \texttt{team\_monitor}, \texttt{weather\_station} e \texttt{debug\_log}.

\subsection*{Passo 0 -- Configurazione e compilazione}
Il comando \texttt{./configure} effettua la configurazione del prototipo in base al proprio sistema. Qualora la procedura automatica non andasse a buon fine, è possibile definire le seguenti variabili d'ambiente:
\begin{itemize}
\item \texttt{ERLC}: percorso del compilatore \Erlang{};
\item \texttt{ERLCFLAGS}: \textsl{flags} per il compilatore \Erlang{};
\item \texttt{ERL}: percorso dell'interprete \Erlang{};
\item \texttt{PYTHON}: percorso dell'interprete \Python{};
\item \texttt{PYTHONPATH}: eventuali percorsi addizionali in cui cercare i moduli \Python{};
\item \texttt{PYRCC}: percorso dell'eseguibile \texttt{pyrcc4} di \textsl{PyQt4};
\item \texttt{PYUIC}: percorso dell'eseguibile \texttt{pyuic4} di \textsl{PyQt4}.
\end{itemize}
La compilazione può dunque essere eseguita tramite l'usuale comando \texttt{make}.

\subsection*{Passo 1 -- Il file \texttt{.hosts.erlang}}
Per prima cosa bisogna inserire i nomi degli \textit{hosts} che faranno parte del sistema nel file \texttt{.hosts.erlang} posto nella medesima \textit{directory} degli script di avvio del prototipo. Ovviamente ciascun \textit{host} deve essere in grado di risolvere il nome di ogni altro \textit{host} del sistema ad un indirizzo IP, ad esempio tramite DNS o configurando opportunamente il file \texttt{/etc/hosts}.

\subsection*{Passo 2 -- Control Panel}
\`E necessario eseguire in un terminale lo \textit{script} \texttt{start\_control\_panel} e inserire i dati richiesti dell'interfaccia grafica.
Poiché in tale interfaccia viene richiesto il percorso dei file di configurazione per la simulazione, è necessario che tali file siano raggiungibili dall'elaborare in cui viene eseguito lo script.

\subsection*{Passo 3 -- Node Configurator}
Per fare in modo che diversi elaboratori partecipino alla simulazione è necessario che in ognuno di questi venga eseguito \texttt{start\_node\_configurator}, indicando quante delle diverse componenti del sistema si ha la possibilità di ospitare.

Una volta che la disponibilità indicata dai diversi nodi distribuiti sarà sufficiente ad ospitare la simulazione, sarà possibile completare la fase di inizializzazione del prototipo premendo il tasto \textsl{Bootstrap}, precedentemente disabilitato, in \textsl{Control Panel}.

\subsection*{Passo 4 -- Interfacce grafiche}
Per avviare le interfacce grafiche è necessario inizializzare la variabile d'ambiente \texttt{FORSE\_NS} con l'\textit{hostname} dell'elaboratore presso il quale è stata avviata la componente \textsl{Control Panel}, qualora esso non coincida con l'\textit{host} locale.

Per poter avviare la simulazione è necessario che almeno una componente \texttt{race\_info} sia avviata in un elaboratore in rete con quelli che ospitano il sistema precedentemente avviato. L'avvio di tale componente avviene tramite l'esecuzione di \texttt{start\_race\_info}.

L'avvio di \texttt{team\_monitor} è opzionale e può essere effettuato tramite lo \textit{script} \texttt{start\_team\_monitor}. Ogni istanza di tale componente viene associata ad una sola scuderia, a seconda della scelta compiuta dall'utente all'avvio di ciascuna istanza.

L'avvio di \texttt{weather\_station} è opzionale e può essere effettuato tramite lo \textit{script} \texttt{start\_weather\_station}.

L'avvio di \texttt{debug\_log} è opzionale e può essere effettuato tramite lo \textit{script} \texttt{start\_logger}.


\chapter{Glossario}

\begin{tabularx}{\textwidth}{lX}
\term{Intermedio cronometrico}{Segmento di lunghezza nulla in corrispondenza del quale vengono raccolti dati sulle prestazioni dell'auto. Anche il traguardo è considerato un segmento di questo tipo.}
\term{Processo}{Un singolo flusso di controllo all'interno di un programma.}
\term{Segmento}{L'unità di spazio più piccola e indivisibile che costituisce il tracciato, utilizzata per la rappresentazione interna dello stesso.}
\term{Settore}{Porzione di tracciato che presenta caratteristiche fisiche costanti per tutta la sua lunghezza. Definito dall'utente in fase di configurazione.}
\term{Simulation speed}{Fattore numerico che serve ad impostare la velocità con la quale la simulazione evolve.}
\term{Tabella di preelaborazione}{Struttura dati riferita ad un'auto in cui sono contenute le triple (ID segmento, velocità massima, velocità massima con \textit{pit stop}) per ogni segmento della pista.}
\end{tabularx}


\chapter{Diario delle modifiche}

\section{Dalla versione 1.0 alla 1.1}

\subsection*{Modifiche alla relazione}
\begin{itemize}
\item Aggiunta la definizione di ``processo'' nel glossario.
\end{itemize}

\subsection*{Modifiche al prototipo}
\begin{itemize}
\item
\end{itemize}
