\documentclass[11pt,a4paper]{report}
\usepackage[utf8]{inputenc}
\usepackage[italian]{babel}
\usepackage[pdftex]{hyperref}
\usepackage{tabularx}

\hypersetup{
	breaklinks,
	colorlinks,
	linkcolor=blue,
	pdftitle={Relazione},
	pdfsubject={Relazione sul progetto di Sistemi Concorrenti e Distribuiti},
	pdfauthor={Daniele Battaglia \& Davide Pesavento},
	pdfcreator={TeXLive-2008},
	pdfproducer={TeXLive-2008}
}

\title{Relazione sul progetto}
\author{Daniele Battaglia\\\url{dbat.fk@gmail.com}
	\and Davide Pesavento\\\url{davidepesa@gmail.com}}
\date{}

\newcommand{\term}[2]{\textbf{#1} & #2 \\}

\begin{document}

\maketitle

\tableofcontents

\clearpage
\chapter{Introduzione}
La presente relazione, assieme al software \textsl{FORSE}, è stata realizzata come prova d'esame per il corso "Sistemi Concorrenti e Distribuiti".

I requisiti considerati in fase di progettazione sono stati ricavati da quelli proposti dal docente (\url{http://www.math.unipd.it/~tullio/SCD/2008/Progetto.html})
secondo le clausole di partecipazione di livello 3.

\textsl{Formula One Race Simulation Engine} è realizzato utilizzando i linguaggi \textsl{Erlang} e \textsl{Python}. Per la distribuzione dell'applicazione abbiamo
usato il protocollo di distribuzione Erlang, supportato nativamente dall'omonimo linguaggio e implementato dalla libreria \textsl{Twotp} per il linguaggio
\textsl{Python}.
Per la creazione di interfacce grafiche è stata utilizzata la libreria \textsl{Qt4}.

Nelle fasi di progettazione e realizzazione del prototipo è stata posta maggiore attenzione riguardo le tematiche inerenti distribuzione e concorrenza, nozioni
centrali del corso, a scapito della parte riguardante la correttezza della simulazione da un punto di vista fisico. Abbiamo comunque cercato di includere
gli elementi essenziali della dinamica di una gara automobilistica anche se con notevoli semplificazioni.

INSERIRE OVERVIEW DELLA RELAZIONE

\chapter{Requisiti}
\begin{enumerate}
    \item Il sistema deve essere composto da più entità concorrenti e distribuite su una rete.
    \item La simulazione evolve in modo deterministico.
    \item Il circuito è suddiviso in settori ed è configurabile dall'utente tramite file di configurazione. Il file contiene le seguenti informazioni:
    \begin{itemize}
        \item[--] tipo di settore (rettilineo, curvilineo, entrata e uscita dai box, intermedio cronometrico, traguardo);
        \item[--] lunghezza e larghezza del segmento;
        \item[--] se il settore è una curva, direzione (destra o sinistra) e raggio di curvatura;
        \item[--] pendenza del suolo;
        \item[--] tempo atmosferico iniziale;
        \item[--] eventuali variazioni del tempo atmosferico nel corso della gara.
    \end{itemize}
    \item Insieme configurabile di concorrenti aventi le seguenti caratteristiche:
    \begin{itemize}
	\item[--] numero identificativo;
	\item[--] nome;
        \item[--] una scuderia di appartenenza;
        \item[--] esperienza/bravura;
        \item[--] peso;
        \item[--] vettura utilizzata.
    \end{itemize}
    \item Parametri di configurazione delle auto:
    \begin{itemize}
        \item[--] capienza del serbatoio;
	\item[--] quantità di carburante presente nel serbatoio a inizio gara;
        \item[--] tipo di pneumatici montato ad inizio gara (da asciutto, intermedio, da bagnato);
        \item[--] potenza del motore;
        \item[--] potenza dei freni;
        \item[--] peso a secco.
    \end{itemize}
    \item Parametri di configurazione di una competizione:
    \begin{itemize}
        \item[--] numero totale di giri da effettuare nella gara;
	\item[--] ordine di partenza dei piloti nella griglia di partenza;
    \end{itemize}
    \item Sistema di controllo (\textit{logging}):
    \begin{itemize}
        \item[--] un pannello generale con posizione dei concorrenti in gara, tempo attuale sul giro e tempo migliore sul giro;
        \item[--] situazione atmosferica della pista;
        \item[--] un pannello per ciascuna scuderia che riporta i parametri tecnici rilevanti delle vetture dei propri piloti (carburante residuo, usura dei pneumatici e loro tipo) nonché i tempi di percorrenza per settore di pista e totali.
    \end{itemize}
\end{enumerate}

%\chapter{\textit{Brainstorming}}
%Finora abbiamo preso in considerazione tre potenziali soluzioni.

%\section*{Soluzione 1}
%Vetture $\Rightarrow$ entità attive \\
%Segmenti del tracciato $\Rightarrow$ entità passive \\

%Per spostarsi da un segmento $S_1$ ad un altro segmento $S_2$ adiacente e successivo a $S_1$, una vettura deve assicurarsi che almeno una corsia di $S_2$ raggiungibile da $S_1$ sia libera, quindi dovrà già essere entrata nella regione protetta di una corsia di $S_2$. Il passaggio sarà completo quando la vettura esce dalla regione protetta di $S_1$ (e questo è un problema non banale). L'entità ``segmento'' espone solo un canale tipato. Le vetture dovranno fornire il numero della corsia da cui provengono e la logica che realizza i sorpassi è implementata nei segmenti effettuando una \texttt{requeue} presso la corsia appropriata.

%\section*{Soluzione 2}
%Segmenti del tracciato $\Rightarrow$ entità attive \\
%Vetture $\Rightarrow$ oggetti che vengono scambiati come parametri \\

%Ciascuna corsia di ciascun segmento è internamente suddivisa in tanti piccoli pezzi, ciascuno dei quali è dotato di una coda contenente le vetture in transito, ordinate per tempo di percorrenza crescente. L'entità ``segmento'' è in ascolto su un canale tipato asincrono, in attesa che il segmento precedente gli invii una nuova vettura. L'attesa prevede un \textit{timeout} pari al minimo tempo di percorrenza di tutte le vettura già accodate su ogni sottosegmento di ogni corsia. Non appena scade il \textit{timeout}, tutte le vetture con tempo di percorrenza pari al \textit{timeout} vengono fatte avanzare al sottosegmento successivo oppure inviate al segmento successivo. Come gestire canali e sorpassi? Usare un \textit{thread pool} per le vetture?

%\section*{Soluzione 3}
%Vetture $\Rightarrow$ entità attive \\
%Segmenti del tracciato $\Rightarrow$ entità passive \\

%Dopo aver calcolato il tempo di uscita della vettura dal segmento $S_i$, esso verrà immediatamente comunicato al segmento $S_{i+1}$ che lo inserirà in un'apposita struttura dati. I canali di ingresso in un settore hanno delle guardie che permettono l'accesso alle vetture solo nell'ordine prestabilito.

\chapter{Architettura del sistema}
Diagramma con l'architettura del sistema ad alto livello e parte introduttiva, serve una descrizione funzionale delle varie parti mettendo in rilievo
come l'utente possa interagire con il sistema. 
\section{Scheduler}
\section{Track}
\section{Car}
\section{Team}
\section{Event Dispatcher}
\section{Weather}
\section{Observers FIXME}
Scegliere un titolo migliore per la sezione e descrivere gli observers una volta decisi e implementati.
\section{Avvio e Terminazione}
Infrastruttura e meccanismi di avvio e terminazione del sistema.
\section{Alternative?}
Eventualmente cassare questa sezione

\chapter{Dettagli di progettazione}
\section{Tecnologie utilizzate}
Breve descizioni dei principi base ed eventuali problemi riscontrati.
\section{Dinamiche della competizione}
\subsection{Partenza}
Descrizione della logica che gestisce la partenza delle auto all'avvio della competizione.
\subsection{Percorrenza di un Segmento}
Descrizione dell'algoritmo
\subsection{Intermedi Cronometrici}
\subsection{Pit Lane}
Descrizione dell'implementazione delle regole di accesso ai box, struttura della pista e algoritmo di sosta e rifornimento dell'auto.
\subsection{Arrivo}
Termiazione dei processi car
\subsection{Interazione con l'Utente}

\section{Distribuzione}
Cosa può essere eseguito in modo distribuito, implicazioni della distribuzione di tali componenti, name server...
\section{Correttezza Temporale}
Considerazioni sulla correttezza del tempo di percorrenza, controllo della concorrenza, algoritmo di sorpasso.
Rallentamento della simulazione non influenza i tempi di gara (scheduler).
\appendix

\chapter{Glossario}

\begin{tabularx}{\textwidth}{lX}
\term{Intermedio cronometrico}{}
\term{Segmento}{L'unità di spazio più piccola e indivisibile che costituisce il tracciato, utilizzata per la rappresentazione interna dello stesso.}
\term{Settore}{Porzione di tracciato che presenta caratteristiche fisiche costanti per tutta la sua lunghezza. Definito dall'utente in fase di configurazione.}
\end{tabularx}

\end{document}
