\chapter{Introduzione}
La presente relazione, unitamente al software denominato \textsl{``Formula One Race Simulation Engine''} (\textsl{FORSE}), è stata realizzata come prova d'esame per il corso ``Sistemi Concorrenti e Distribuiti''.

I requisiti considerati in fase di progettazione sono stati ricavati da quelli proposti dal docente\footnote{\url{http://www.math.unipd.it/~tullio/SCD/2008/Progetto.html}} secondo le clausole di partecipazione di livello 3.

\textsl{FORSE} è stato realizzato utilizzando i linguaggi di programmazione \Erlang{} e \Python{}. Per la distribuzione dell'applicazione abbiamo
usato il protocollo di distribuzione \Erlang{}, supportato nativamente dall'omonimo linguaggio e implementato dalla libreria \textsl{TwOTP} per il linguaggio
\Python{}.
Per la creazione di interfacce grafiche è stata utilizzata la libreria \textsl{Qt}.

Nelle fasi di progettazione e realizzazione del prototipo è stata posta maggiore attenzione riguardo le tematiche inerenti distribuzione e concorrenza, nozioni
centrali del corso, a scapito della parte riguardante la correttezza della simulazione da un punto di vista fisico. Abbiamo comunque cercato di includere
gli elementi essenziali della dinamica di una gara automobilistica anche se con alcune semplificazioni.

Nella prima parte del documento verranno esposti i requisiti espliciti, ricavati dalle richieste del docente, ed impliciti, estratti dal contesto reale che il software \textsl{FORSE} deve simulare.
In seguito verranno elencati i problemi intrinseci che la progettazione del sistema deve affrontare ed essi verranno analizzati alla luce delle conoscenze acquisite durante il corso.
Nella seconda parte della relazione verrà illustrata la nostra soluzione, prima a livello progettuale/architetturale e successivamente a livello di implementazione, mostrando come i problemi precedentemente individuati vengano risolti in modo soddisfacente dal prototipo.
