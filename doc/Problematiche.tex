\chapter{Problematiche di progettazione}

In questo capitolo verranno individuati i principali problemi incontrati nella progettazione del sistema, sia a livello di coordinazione delle componenti distribuite, sia a livello di meccaniche di simulazione.

\subsection*{Avvio del sistema}
La simulazione non può iniziare finché tutte le componenti necessarie non siano state avviate correttamente e siano pronte all'esecuzione. Per esempio la gara non deve iniziare prima che tutte le auto e le scuderie che vi partecipano siano pronte e correttamente inizializzate, altrimenti si potrebbero avere delle inconsistenze nella simulazione.

\`E necessario avere a dispozione un meccanismo di sincronizzazione che implementi il concetto di barriera parallela, in modo che ogni processo, una volta raggiunto il punto di \textsl{rendez-vous}, non possa procedere nella sua esecuzione finché tutti i partecipanti alla sincronizzazione non siano giunti al "medesimo" punto. Una volta che la sincronizzazione ha avuto successo tutti i processi convolti possono procedere nella loro esecuzione.


\subsection*{Accesso alla pista}
Le modifiche alle condizioni della pista devono essere eseguite in mutua esclusione per evitare inconsistenze nel suo stato interno, è necessario quindi che i processi che accedono alle informazioni sulla pista eseguano in modo ordinato e solo quando previsto dalla logica di simulazione. In particolare non deve accadere che due auto effettuino contemporaneamente uno spostamento oppure che vengano modificate le condizioni atmosferiche mentre un'auto si sta muovendo sul tracciato.
La concorrenza deve quindi essere controllata per evitare queste situazioni di \textit{race condition}.

\subsection*{Sorpassi coerenti}
Quando due auto si trovano a percorrere lo stesso segmento di pista devono interagire in modo realistico. Devono essere quindi chiare e verosimili le condizioni che permettono ad un'auto di sorpassarne un'altra, per evitare che si verifichino sorpassi dove la pista o la presenza di altre auto non lo consentano.

\subsection*{Comunicazione di sosta}
Un'auto effettua il rifornimento solo quando richiesto da un attore esterno. Questo attore può essere la propria scuderia o l'utente.

La scuderia indica all'auto durante quale giro di gara effettuare la sosta in base alle informazioni raccolte sulle \textit{performance} dell'auto stessa, di conseguenza è possibile che, al variare delle condizioni della pista, la scuderia calcoli una nuova strategia. Nel caso di un cambiamento delle condizioni atmosferiche, per esempio, la scuderia può decidere di far rientrare ai \textit{box} un'auto al fine di montare dei pneumatici adeguati alle nuove condizioni. Risulta quindi evidente che il messaggio usato dalla scuderia per comunicare all'auto in che giro effettuare la sosta successiva può essere inviato più volte e l'auto deve tenere in considerazione solo quello più recente.

L'utente può invece forzare la sosta immediata di un'auto ai \textit{box}. L'interazione con l'utente avviene in modo asincrono rispetto alla simulazione e deve avere la precedenza sulle direttive provenienti dalla scuderia.

Le situazioni da evitare sono quindi due:
\begin{itemize}
\item Un'auto effettua una sosta ai \textit{box} sulla base di informazioni non aggiornate,
\item La richiesta di sosta proveniente dall'utente viene ignorata a causa di una successiva ricezione di direttive diverse dalla scuderia.
\end{itemize}

\subsection*{Variazione condizioni atmosferiche}
Le variazioni alle condizioni atmosferiche sono effettuate dal sistema in modo sincrono alla competizione nel caso in cui siano state previste a livello di configurazione oppure possono avvenire a seguito di un intervento dell'utente in modo asincrono rispetto alla competizione. Il sistema deve quindi permettere di effettuare queste modifiche evitando i problemi di \textit{race condition} precedentemente evidenziati.

\subsection*{Realismo}
Poiché il prototipo deve simulare una situazione reale, si è reso necessario trovare un \textit{trade-off} adeguato tra la semplicità di implementazione e l'accuratezza della simulazione. Questo vale sia dal punto di vista delle leggi fisiche a cui sono soggette le auto nel percorrere un segmento, sia dal punto di vista della rappresentazione della pista e delle sue regole di percorrenza. Nella realtà per esempio due auto appartenenti alla stessa scuderia non possono effettuare il rifornimento in contemporanea e devono rispettare i limiti di velocità imposti dal regolamento quando si trovano nella \textit{pit lane}.

\subsection*{Fine della competizione}
Il sistema deve essere in grado di riconoscere la fine della competizione ovvero quando l'ultima auto in gara taglia il traguardo. In risposta a questo evento il sistema deve reagire in modo opportuno, mettendosi in uno stato che impedisca alla simulazione di procedere inutilmente.

Deve essere quindi previsto un modo per comunicare a tutte le componenti del sistema di terminare la propria esecuzione, nel caso in cui non servano più, oppure di mettersi in uno stato di sospensione per permettere all'utente di consultare i dati raccolti durante la simulazione.
 
